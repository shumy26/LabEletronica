\documentclass[10pt,twocolumn,letterpaper]{article}

\usepackage{cvpr}
\usepackage{times}
\usepackage{epsfig}
\usepackage{graphicx, color, xcolor}
\usepackage{amsmath}
\usepackage{amssymb}
\usepackage{bm}
\usepackage[utf8]{inputenc}
\usepackage{fancyhdr}
\usepackage{graphicx}
\usepackage[brazil]{babel}
\usepackage{lipsum}
\usepackage{float}
\setlength{\headheight}{1.5cm}

\fancypagestyle{plain}
\lhead{\includegraphics[width=5cm,height=2cm]{logoFT.png}}
\rhead{\includegraphics[width=5cm,height=2cm]{logoUnB.png}}

\renewcommand{\headrulewidth}{1pt}%
}

% Changing the caption and 'References' names to Portuguese.
\addto\captionsenglish{
  \renewcommand{\figurename}{Figura}
  \renewcommand{\tablename}{Tabela}
  \renewcommand{\refname}{Referências bibliográficas}
}

% to-do: Using fontenc with T1 font encoding allows \hyphenation to take words with accents
% as arguments. However, fontenc with T1 mess up with words containing 'ã' in all \subsection{}
% commands.
% If someone knows how to fix it, tell me!
%
%\usepackage[T1]{fontenc}

% Put the words not correctly hyphenated down here. Words with accents must
% be manually hyphenated in the text itself using the '\-' separator. See the examples
% throughout this template.
\hyphenation{co-e-ren-te u-sa-do de-li-mi-ta-do-res a-cer-ca va-lo-res cor-res-pon-den-te re-fe-ren-ci-a-das co-lu-na co-lu-nas em-bo-ra u-ti-li-da-de pro-ce-di-men-to ex-pe-ri-men-to fi-gu-ras}

% If you comment hyperref and then uncomment it, you should delete
% egpaper.aux before re-running latex.  (Or just hit 'q' on the first latex
% run, let it finish, and you should be clear).
\usepackage[breaklinks=true,bookmarks=true]{hyperref}

\cvprfinalcopy % *** Uncomment this line for the final submission

%\def\cvprPaperID{****} % *** Enter the CVPR Paper ID here
%\def\httilde{\mbox{\tt\raisebox{-.5ex}{\symbol{126}}}}

% Pages are numbered in submission mode, and unnumbered in camera-ready
% \ifcvprfinal\pagestyle{empty}\fi
%\setcounter{page}{1}


\begin{document}
%%%%%%%%% TITLE
\title{Modelo em \LaTeX\ de Relatório para a Disciplina de Laboratório de Eletrônica}

\author{Aluno 1\\
% To save space, use either the email address or home page, not both
{\tt\small aluno1@aluno.unb.br}\\
% Matrícula do primeiro autor
{\tt\small 00/0000000}\\
% Turma
{\tt\small Turma A/D}
\and
Aluno 2\\
{\tt\small aluno2@aluno.unb.br}\\
% Matrícula do segundo autor
{\tt\small 00/0000000}\\
% Turma
{\tt\small Turma A/D}
\and
Aluno 3\\
{\tt\small aluno3@aluno.unb.br}\\
% Matrícula do terceiro autor, se houver
{\tt\small 00/0000000}\\
% Turma
{\tt\small Turma A/D}
}

\maketitle
%\thispagestyle{empty}
\section{Procedimento experimental}

O experimento 1 consiste em uma familiarização com o software LTSpice para simulações de circuitos. A primeira simulação é de um divisor de corrente com uma fonte de 0.31mA e 3 resistores, com $R_1 = 26k\Omega, R_2 = 28k\Omega, R_3 = 1k\Omega$

\begin{figure}
\includegraphics[scale=0.4]{\LTSpice\Exp1}
\end{figure}
%-------------------------------------------------------------------------
\section{Resultados e análises}

A seção {\em Resultados e análises} deve apresentar os resultados obtidos após a execução do experimento em laboratório, bem como as análises dos mesmos, comparando-os com os valores obtidos nas simulações, dados na seção anterior, e com aqueles calculados segundo a teoria.  

%------------------------------------------------------------------------
\section{Conclusão}

A seção {\em Conclusão} deve conter as conclusões alcançadas ao fim do experimento, as lições aprendidas, os problemas não resolvidos e as possíveis soluções que poderiam ser implementadas em experimentos futuros. Os alunos devem finalizar tratando do alcance, ou não, dos objetivos e da utilidade do experimento tendo em vista as aplicações e o contexto no qual se insere o tema do roteiro, os quais devem ter sido apresentados na seção {\em Introdução} (pré-relatório).

{\small
\bibliography{egbib}
\bibliographystyle{ieee_fullname}
}

\end{document}
